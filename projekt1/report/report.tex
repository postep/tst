\documentclass[]{article}
\usepackage{polski}
\usepackage[utf8]{inputenc}
\usepackage{amsmath}
\usepackage{graphicx}
\usepackage{geometry}
\usepackage{float}

\geometry{legalpaper, margin=0.6in}

%opening
\title{TST - Projekt 1}
\author{Jakub Postępski}

\begin{document}

\maketitle


\section{Wprowadzenie}
Przyjmijmy macierz $A_{2x2}$.

Macierz posiada, dla $i = {1, 2}$:
\begin{itemize}
	\item wartości własne $\lambda_i$
	\item wektory własne $\vec{v_i}$ 
\end{itemize}

W zadaniu interesuje nas dynamika ciągu:
\[x^{(t)} = A^tx_0 \]

\section{Wartości własne}

Podstawowym i najważniejszym równaniem jest:
\[ Ax = \lambda x\]

Zbiór rozwiązań to widmo: \[\sigma(A) = \{ \lambda_i \in \  \} \]
W ten sposób szukamy wartości dla których macierz $A$ zachowa się jak przekształcenie liniowe (skalar). Dla takich wartości macierz tylko powiększa lub zmniejsza zadany wektor, ale nie zależności pomiędzy składowymi wektora. 

Z równania wynika, że wartości własne decydują o tym jakim przekształceniem będzie macierz $A$

\section{Wektory własne}
Są wektorami rozpinającymi bazę którą tworzy przekształcenie macierzy A. Zestaw takich wektorów nie jest jednoznaczny. Jedną przestrzeń można rozpiąć różnymi zestawami wektorów. 

\section{Konstruowanie macierzy o zadanych parametrach}
Odpowiedź wynika z:
\[ AV = VD \]

Dostajemy:
\[A = VDV^-1\]


\end{document}
