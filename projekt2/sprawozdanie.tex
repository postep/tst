\documentclass[a4paper]{article}
\usepackage{polski}
\usepackage[utf8]{inputenc}
\usepackage{amsmath}
\usepackage{graphicx}
\usepackage{caption}
\usepackage{subcaption}
\usepackage{geometry}
\usepackage{float}


\title{TST - Projekt 2}
\author{Jakub Postępski}

\begin{document}

\maketitle


\section{Bieguny}
\begin{itemize}

	\item przy lokowaniu biegunów poza układem jednostowym układ "rozjeżdza się".
	\item zwiększanie odległości od zera lokowanych biegunów minimalnie przyspiesza regulację, ale zwiększa wpływ szumów
	\item przy lokowaniu ujemnych biegunów układ obserwatora dużo gorzej reaguje na szumy
\end{itemize}

\subsection{Bieguny obserwatora}
Można stworzyć obserwator złożony z dwuczęściowego (odzielnie obserwujemy macierz $\Phi$ i $\Phi_w$). Wtedy nie bierzemy pod uwagę wpływu $Phi_{xw}$. Drugą opcją jest zaprojektowanie obserwatora dla całego układu od razu co daje mniejsze błędy estymacji.

\subsection{Bieguny $K_x$}
Z równania 

\[det(zI - A + BK_x)\]

daje się znaleźć tylko liniową zależność ponieważ:

\[rank(W_c) < rank(\Phi)\] 

więc układ nie jest sterowalny. Ustawienie dużych wartości wektora $K_x$powoduje większe przenoszenie szumów przez większe wartości sterowania.

\subsection{Bieguny $K_w$}
Po przekształceniu równania 

\[ \hat{x}(x+1) = \Phi x(t) + \Phi_{xw}w(t) + \Gamma u(t) \]

i podstawieniu 

\[ U(t) = -K_xx(t)-K_ww(t) \]

dostajemy:

\[ w(t) = (\Phi_{xw} - \Gamma K_w) \Rightarrow K_w = \Gamma ^{-1} \Phi_{xw}\]

ale wymiary macierzy się nie zgadzają. Żeby znaleźć coś podobnego do odwrotności zastosowałem pseudoodwrotność macierzy.

\section{Sterowanie}
\begin{itemize}
\item widać wyraźne piki sterowania przy zmianie wartości zadanej 
\item nie ma eliminacji uchybu statycznego. Wartość uchybu statycznego zależy od $K_c$
\item przy niektórych symulacjach widać jak nieskompensowany szum zaczyna wprowadzać coraz większe oscylacje do obserwatora.
\end{itemize}

\section{Wartość $K_c$}
Układ bez sterowania zewnętrznego zbiega do zera. Po dodaniu do sterowania wartości zadanej układ powinien zbiegać do niej. Zależy nam na tym aby w stanie ustalonym (możemy pominąć indeksy czasu i wpływ szumów):

\[ y = Cx = u_c\]

Po podstawieniu tam gdzie się da $u_c=y$ i zastosowaniu macierzy pseudoodwrotnej dostajemy:

\[ K_c = \Gamma^{-1} (1-Phi+\Gamma K_x)C^{-1}\]

i wtedy moglibyśmy wyliczyć $K_c$, gdyby nie to, że $K_x$ jest liniowo zależne.

Na zwykłą logikę jeśli $K_c=1$ to układ powinien podążać za $u_c$ i dla symulacji faktycznie ma to miejsce. 

Można na to wszystko spojrzeć w przewrotny sposób. Można odwrócić powyższe równanie i wyliczyć wartości $K_x$, przy założeniu $K_c=1$. Przy tak dobranych wartościach szumy są małe i układ reguluje się dobrze.

\end{document}
